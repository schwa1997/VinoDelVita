\section{1. Objectives of the study}
The objective of this project is to develop a comprehensive web application that supports vine disease management for the firm's numerous customer companies distributed across various countries. The application aims to facilitate early detection of vine diseases, monitor disease progression, and plan necessary interventions. Integrating this application into the firm's service offerings, will enhance the quality of vineyard management and help maintain healthy vineyards.\\

\subsection{1.1 Description of the objectives}
Key Features and Functionality:
\begin{enumerate}
\item Disease Detection and Tracking: The web application will allow users equipped with mobile devices and GPS capabilities to perform visual inspections of vineyards and detect three types of vine diseases (A, B, and C). Users can input the disease type, position, date, and time of detection.

\item Geospatial Visualization: The application will utilize the UTM WGS84 reference system to accurately represent the geographical distribution of vineyard areas and individual vineyards. Geospatial data will be visualized on an interactive map, enabling users to easily locate specific areas and vineyards.

\item Vineyard Management: Users will be able to manage their vineyard data, including area codes, vineyard numbering, year of planting, and vine types. This information will be stored in a structured manner to facilitate effective disease tracking and intervention planning.

\item Treatment Planning and Monitoring: Based on disease detection and other variables, the application will provide recommendations for necessary interventions or treatments. Users can log the treatments applied, which will be associated with specific vineyards. The application will also support remote monitoring of treatment effectiveness using aerial imagery captured by the hexacopter's azimuthal camera.

\item Data Integration: The application will enable users to upload shape files containing geometries and attributes of their vineyard areas and vineyards. The system will process and store this data, ensuring it is accessible for disease monitoring and analysis.

\item User Collaboration: As the application serves multiple customer companies, it will support user accounts with varying levels of access and permissions. Agronomists and farm personnel from different companies can collaborate by sharing disease detection information and treatment outcomes.

\item Reporting and Insights: The application will generate reports summarizing disease occurrences, treatment histories, and intervention effectiveness. These reports will help agronomists and farm managers make informed decisions to optimize vineyard health.

\item Automated Alerts: The application will have the capability to send automated alerts to users when disease detections are recorded, ensuring timely responses and interventions.

\item Scalability and Global Compatibility: The application will be designed to accommodate the firm's widespread customer base, ensuring compatibility with different countries, languages, and regulations.

\item User-Friendly Interface: The web application will feature an intuitive and user-friendly interface, making it accessible to agronomists and farm personnel with varying levels of technical expertise.
\end{enumerate}
By successfully developing and implementing this web application, the agronomist firm aims to provide an indispensable tool for vineyard disease management, fostering healthier vineyards, improving crop yields, and enhancing customer satisfaction. The inclusion of the hexacopter-based remote monitoring service further strengthens the firm's commitment to leveraging advanced technologies for sustainable agricultural practices.

\subsection{1.2 References to the Development plan}
 \begin{enumerate}
     \item Research and system design


At the onset of the project, the primary focus is on defining the project scope and objectives. This involves a thorough understanding of the project's goals, requirements, and expected outcomes. Additionally, the choice of a suitable technology stack for this full-stack project is a critical decision that will impact the development process. The selection of technologies must align with the project's requirements and long-term goals, taking into consideration factors such as scalability, security, and maintainability.

\item Database design

The database design phase involves the creation of the project's database infrastructure. This includes tasks such as creating the database itself, defining the data schema, and creating the necessary tables to store data. Additionally, data insertion processes are established during this phase, ensuring that the system can efficiently handle and manage data.

\item Back-end development

The back-end development phase is dedicated to setting up the server-side of the application. Key activities include configuring the basic infrastructure required for the application to run, such as server setup, database connectivity, and security configurations. Once the foundation is in place, the development team focuses on writing APIs (Application Programming Interfaces) that enable communication between the front-end and back-end components. These APIs serve as the bridge that allows data exchange and functionality execution.

\item Front-end development

In the front-end development phase, the user interface (UI) of the application is designed and implemented. This includes creating visually appealing and user-friendly web pages or mobile app screens. The front-end development team also integrates the APIs developed in the previous phase, ensuring that the user interface can interact with the back-end system seamlessly. 

\item Full-stack development

The full-stack development phase involves comprehensive testing of both the front-end and back-end components to ensure that they work harmoniously together. This phase also includes making any necessary modifications or improvements based on testing results and user feedback. It is crucial to address any bugs, performance issues, or usability concerns during this phase to deliver a robust and user-friendly product.

 \end{enumerate}