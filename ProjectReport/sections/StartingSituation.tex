\section{3. Starting Situation}

\subsection{3.1 Organisational Context (and Business Processes)}
In our organisational context, we are an agronomist firm specializing in vineyard management. Our business processes involve providing consultation services to wineries for disease management, crop health, and vineyard optimization. Currently, our agronomists conduct manual inspections and interventions in response to disease outbreaks, requiring on-site visits and data recording.

\subsection{3.2 Technology}
Our current technology landscape includes basic data recording tools, GPS-enabled mobile devices, and conventional mapping methods. We lack a centralized digital platform for data collection, analysis, and collaboration. There's an opportunity to leverage geospatial technology, data analytics, and web applications to enhance our disease management capabilities.

\subsection{3.3 Existing Data}
We possess historical disease data collected through manual inspections, including disease types, locations, and intervention records. However, this data is fragmented and stored in various formats. Integration and digitization of this data would be valuable for trend analysis and informed decision-making.

\subsection{3.4 Market Analysis}
The agricultural sector, particularly vineyard management, is increasingly embracing technology for improved efficiency and sustainability. There is a growing demand for digital solutions that can provide timely disease detection, data-driven insights, and proactive intervention planning.

\section{3.5 Constraints}

\subsection{3.5.1 Normative}
Our project must adhere to relevant regulations and industry standards for data privacy, security, and environmental impact.

\subsection{3.5.2 Economical}
Our budget for the project is limited and should be allocated efficiently to ensure cost-effectiveness.

\subsection{3.5.3 Organisational}
The project should align with the existing workflows and roles of our agronomists and vineyard management teams.

\subsection{3.5.4 Time}
We aim to have the initial version of the application developed and deployed within 12 months to coincide with the upcoming vineyard planting season.

\subsection{3.5.5 Technological}
Our solution should be compatible with commonly used devices (smartphones, tablets) and web browsers. The chosen technologies should also support future scalability.

\subsection{3.5.6 Recommendation}
Based on the constraints, it is recommended to prioritize a phased development approach that focuses on delivering core features within the specified budget and timeline. Regular communication with stakeholders and agile project management methodologies can help ensure successful development and deployment.
